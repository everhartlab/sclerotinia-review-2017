\section*{References}\label{references}
\addcontentsline{toc}{section}{References}

\hypertarget{refs}{}
\hypertarget{ref-Arnaud-Haond2007-zo}{}
Arnaud-Haond S, Duarte CM, Alberto F, Serrão EA (2007) Standardizing
methods to address clonality in population studies. Molecular Ecology
16:5115--5139

\hypertarget{ref-Atallah2004-es}{}
Atallah ZK, Larget B, Chen X, Johnson DA (2004) High genetic diversity,
phenotypic uniformity, and evidence of outcrossing in \emph{Sclerotinia
sclerotiorum} in the columbia basin of Washington state. Phytopathology
94:737--742

\hypertarget{ref-Atehnkeng2016-qb}{}
Atehnkeng J, Donner M, Ojiambo PS, Ikotun B, Augusto J, Cotty PJ,
Bandyopadhyay R (2016) Environmental distribution and genetic diversity
of vegetative compatibility groups determine biocontrol strategies to
mitigate aflatoxin contamination of maize by \emph{Aspergillus flavus}.
Microbial Biotechnology 9:75--88

\hypertarget{ref-Attanayake2012-mq}{}
Attanayake RN, Porter L, Johnson DA, Chen W (2012) Genetic and
phenotypic diversity and random association of DNA markers of isolates
of the fungal plant pathogen \emph{Sclerotinia sclerotiorum} from soil on a
fine geographic scale. Soil Biology Biochem 55:28--36

\hypertarget{ref-Attanayake2014-uy}{}
Attanayake RN, Tennekoon V, Johnson DA, Porter LD, Río-Mendoza L del,
Jiang D, Chen W (2014) Inferring outcrossing in the homothallic fungus
\emph{Sclerotinia sclerotiorum} using linkage disequilibrium decay.
Heredity 113:353--363

\hypertarget{ref-Bailleul2016-lw}{}
Bailleul D, Stoeckel S, Arnaud-Haond S (2016) RClone: A package to
identify MultiLocus clonal lineages and handle clonal data sets in R.
Methods in Ecology and Evolution 7:966--970. doi:
\href{https://doi.org/10.1111/2041-210x.12550}{10.1111/2041-210x.12550}

\hypertarget{ref-Barari2012-dn}{}
Barari H, Alavi V, Badalyan SM (2012) Genetic and morphological
differences among populations of \emph{Sclerotinina sclerotiorum} by
microsatellite markers, mycelial compatibility groups (MCGs) and
aggressiveness in north iran. Romanian Agricultural Research 29 323--331

\hypertarget{ref-Carbone2001-hi}{}
Carbone I, Kohn LM (2001) Multilocus nested haplotype networks extended
with DNA fingerprints show common origin and fine-scale, ongoing genetic
divergence in a wild microbial metapopulation. Molecular Ecology
10:2409--2422

\hypertarget{ref-Carling1988-xk}{}
Carling DE, Kuninaga S, Leiner RH (1988) Relatedness within and among
intraspecific groups of \emph{Rhizoctonia solani}: A comparison of
grouping by anastomosis and by DNA hybridization. Phytoparasitica
16:209--210

\hypertarget{ref-Chang2014-rn}{}
Chang SW, Jo Y-K, Chang T, Jung G (2014) Evidence for genetic similarity
of vegetative compatibility groupings in \emph{Sclerotinia homoeocarpa}. Plant
Pathology 30:384--396

\hypertarget{ref-Correll1987-hh}{}
Correll JC (1987) Nitrate nonutilizing mutants of \emph{Fusarium
oxysporum} and their use in vegetative compatibility tests.
Phytopathology 77:1640

\hypertarget{ref-Cubeta1997-rr}{}
Cubeta MA, Cody BR, Kohli Y, Kohn LM (1997) Clonality in
\emph{Sclerotinia sclerotiorum} on infected cabbage in eastern North
Carolina. Phytopathology 87:1000--1004

\hypertarget{ref-Derbyshire2017-mx}{}
Derbyshire M, Denton-Giles M, Hegedus D, Seifbarghy S, Rollins J, Kan J
van, Seidl MF, Faino L, Mbengue M, Navaud O, Raffaele S, Hammond-Kosack
K, Heard S, Oliver R (2017) The complete genome sequence of the
phytopathogenic fungus \emph{Sclerotinia sclerotiorum} reveals insights
into the genome architecture of broad host range pathogens. Genome
Biology and Evolution

\hypertarget{ref-Ford1995-wk}{}
Ford EJ, Miller RV, Gray H, Sherwood JE (1995) Heterokaryon formation
and vegetative compatibility in \emph{Sclerotinia sclerotiorum}.
Mycological Research 99:241--247

\hypertarget{ref-Glass2000-cg}{}
Glass NL, Jacobson DJ, Shiu PK (2000) The genetics of hyphal fusion and
vegetative incompatibility in filamentous ascomycete fungi. Annu Rev
Genet 34:165--186

\hypertarget{ref-Glass1992-af}{}
Glass NL, Kuldau GA (1992) Mating type and vegetative incompatibility in
filamentous ascomycetes. Annual Reviews of Phytopathology 30:201--224

\hypertarget{ref-Gordon1992-fs}{}
Gordon TR, Okamoto D (1992) Variation in mitochondrial DNA among
vegetatively compatible isolates of \emph{Fusarium oxysporum}.
Experimental Mycology 16:245--250. doi:
\href{https://doi.org/10.1016/0147-5975(92)90033-n}{10.1016/0147-5975(92)90033-n}

\hypertarget{ref-Goss2015-ue}{}
Goss EM (2015) Genome-enabled analysis of plant-pathogen migration.
Annual Reviews of Phytopathology 53:121--135

\hypertarget{ref-Grubisha2010-ld}{}
Grubisha LC, Cotty PJ (2010) Genetic isolation among sympatric
vegetative compatibility groups of the aflatoxin-producing fungus
aspergillus flavus. Molecular Ecology 19:269--280

\hypertarget{ref-Grunwald2017-wd}{}
Grünwald NJ, Everhart SE, Knaus BJ, Kamvar ZN (2017) Best practices for
population genetic analyses. Phytopathology 107:1000--1010

\hypertarget{ref-Hambleton2002-an}{}
Hambleton S, Walker C, Kohn LM (2002) Clonal lineages of
\emph{Sclerotinia sclerotiorum} previously known from other crops
predominate in 1999-2000 samples from ontario and quebec soybean.
Canadian Journal of Plant Pathology 24:309--315

\hypertarget{ref-Jo2008-ft}{}
Jo Y-K, Chang SW, Rees J, Jung G (2008) Reassessment of vegetative
compatibility of \emph{Sclerotinia homoeocarpa} using
nitrate-nonutilizing mutants. Phytopathology 98:108--114

\hypertarget{ref-Kamvar2015-ff}{}
Kamvar ZN, Brooks JC, Grünwald NJ (2015) Novel R tools for analysis of
genome-wide population genetic data with emphasis on clonality.
Frontiers in Genetics 6:208. doi:
\href{https://doi.org/10.3389/fgene.2015.00208}{10.3389/fgene.2015.00208}

\hypertarget{ref-Kamvar2017-cl}{}
Kamvar ZN, Sajeewa Amaradasa B, Jhala R, McCoy S, Steadman JR, Everhart
SE (2017) Population structure and phenotypic variation of
\emph{Sclerotinia sclerotiorum} from dry bean (\emph{Phaseolus
vulgaris}) in the United States. PeerJ 5:e4152

\hypertarget{ref-Kohli1998-hh}{}
Kohli Y, Kohn LM (1998) Random association among alleles in clonal
populations of \emph{Sclerotinia sclerotiorum}. Fungal Genetics and
Biology 23:139--149

\hypertarget{ref-Kohli1992-pe}{}
Kohli Y, Morrall RAA, Anderson JB, Kohn LM (1992) Local and
Trans-Canadian clonal distribution of \emph{Sclerotinia sclerotiorum} on
canola. Phytopathology 82:875

\hypertarget{ref-Kohn1990-po}{}
Kohn LM, Carbone I, Anderson JB (1990) Mycelial interactions in
\emph{Sclerotinia sclerotiorum}. Experimental Mycology 14:255--267

\hypertarget{ref-Kohn1991-wq}{}
Kohn LM, Stasovski E, Carbone I, Royer J, Anderson JB (1991) Mycelial
incompatibility and molecular markers identify genetic variability in
field populations of \emph{Sclerotinia sclerotiorum}. Phytopathology 81:480

\hypertarget{ref-Lehner2017-ny}{}
Lehner MS, Mizubuti ESG (2017) Are \emph{Sclerotinia sclerotiorum}
populations from the tropics more variable than those from subtropical
and temperate zones? Tropical Plant Pathology 42:61--69

\hypertarget{ref-Lehner2017-mm}{}
Lehner MS, Paula Júnior TJ de, Del Ponte EM, Mizubuti ESG, Pethybridge
SJ (2017) Independently founded populations of \emph{Sclerotinia
sclerotiorum} from a tropical and a temperate region have similar
genetic structure. PLoS One 12:e0173915

\hypertarget{ref-Lehner2015-oj}{}
Lehner MS, Paula Júnior TJ, Hora Júnior BT, Teixeira H, Vieira RF,
Carneiro JES, Mizubuti ESG (2015) Low genetic variability in \emph{Sclerotinia
sclerotiorum} populations from common bean fields in Minas Gerais state,
Brazil, at regional, local and micro-scales. Plant Pathology 64:921--931

\hypertarget{ref-Leslie1993-hj}{}
Leslie JF (1993) Fungal vegetative compatibility. Annual Reviews of
Phytopathology 31:127--150

\hypertarget{ref-Leslie1996-di}{}
Leslie JF, Zeller KA (1996) Heterokaryon incompatibility in fungi---more
than just another way to die. J Genet 75:415--424

\hypertarget{ref-Liu1996-dr}{}
Liu Y-C, Cortesi P, Double ML, MacDonald WL, Milgroom, Michael G (1996)
Diversity and multilocus genetic structure in populations of
\emph{Cryphonectria parasitica}. Phytopathology 86:1344--1351

\hypertarget{ref-Malvarez2007-jo}{}
Malvárez G, Carbone I, Grünwald NJ, Subbarao KV, Schafer M, Kohn LM
(2007) New populations of \emph{Sclerotinia sclerotiorum} from lettuce
in California and peas and lentils in Washington. Phytopathology
97:470--483

\hypertarget{ref-McDonald1997-ob}{}
McDonald BA (1997) The population genetics of fungi: Tools and
techniques. Phytopathology 87:448--453

\hypertarget{ref-Micali2003-li}{}
Micali CO, Smith ML (2003) On the independence of barrage formation and
heterokaryon incompatibility in \emph{Neurospora crassa}. Fungal Genetics
Biology 38:209--219

\hypertarget{ref-Milgroom1996-we}{}
Milgroom MG (1996) Recombination and the multilocus structure of fungal
populations. Annual Reviews of Phytopathology 34:457--477

\hypertarget{ref-Milgroom2003-mu}{}
Milgroom MG, Peever TL (2003) Population biology of plant pathogens: The
synthesis of plant disease epidemiology and population genetics. Plant
Disease 87:608--617

\hypertarget{ref-Neigel1983-gt}{}
Neigel JE, Avise JC (1983) Clonal diversity and population structure in
a reef-building coral, \emph{Acropora cervicornis}: Self-recognition
analysis and demographic interpretation. Evolution 37:437--453

\hypertarget{ref-Papaioannou2014-pe}{}
Papaioannou IA, Typas MA (2014) Barrage formation is independent from
heterokaryon incompatibility in \emph{Verticillium dahliae}. European Journal
of Plant Pathology 141:71--82

\hypertarget{ref-Parks1993-nv}{}
Parks JC, Werth CR (1993) A study of spatial features of clones in a
population of bracken fern, \emph{Pteridium aquilinum}
(dennstaedtiaceae). American Journal of Botany 80:537

\hypertarget{ref-Perkins1988-mt}{}
Perkins DD (1988) Main features of vegetative incompatibility in
\emph{Neurospora}. Fungal Genetic Reports 35:44

\hypertarget{ref-Phillips2002-pq}{}
Phillips DV, Carbone I, Gold SE, Kohn LM (2002) Phylogeography and
genotype-symptom associations in early and late season infections of
canola by \emph{Sclerotinia sclerotiorum}. Phytopathology 92:785--793

\hypertarget{ref-Prugnolle2010-yb}{}
Prugnolle F, De Meeus T (2010) Apparent high recombination rates in
clonal parasitic organisms due to inappropriate sampling design.
Heredity 104:135--140

\hypertarget{ref-Puhalla1985-bq}{}
Puhalla JE (1985) Classification of strains of \emph{Fusarium oxysporum} on the
basis of vegetative compatibility. Can J Bot 63:179--183

\hypertarget{ref-Schafer2006-ph}{}
Schafer MR, Kohn LM (2006) An optimized method for mycelial
compatibility testing in \emph{Sclerotinia sclerotiorum}. Mycologia
98:593--597

\hypertarget{ref-Sirjusingh2001-sq}{}
Sirjusingh C, Kohn LM (2001) Characterization of microsatellites in the
fungal plant pathogen, \emph{Sclerotinia sclerotiorum}. Molecular
Ecology Notes 1:267--269

\hypertarget{ref-Strom2016-di}{}
Strom NB, Bushley KE (2016) Two genomes are better than one: History,
genetics, and biotechnological applications of fungal heterokaryons.
Fungal Biol Biotechnol 3:4

\hypertarget{ref-Wu2017-dx}{}
Wu S, Cheng J, Fu Y, Chen T, Jiang D, Ghabrial SA, Xie J (2017)
Virus-mediated suppression of host non-self recognition facilitates
horizontal transmission of heterologous viruses. PLoS Pathogens
13:e1006234

\end{document}
